
\section{Integration}
\label{sec:quadpack}
\secauthor{Travis Oliphant, Oliphant.Travis@altavista.net}

\subsection{Univariate Integration}
Consider a function $f:\R \rightarrow \R$ with values
denoted by $f(x)$ with a possible weight function $w:\R \rightarrow \R$
with values $w(x)$.  Suppose we are interested in computing the value
of the definite integral
\[
\int_a^b f(x) w(x) dx.
\]
\code{Multipack} (\code{quadpack}) provides a function
\code{quad} for integrating such univariate functions possibly multiplied
by various weights over finite and infinite intervals.  It is a Python-C
wrapper of the double-precision FORTRAN integrators \code{qags},
\code{qagi}, \code{qagp}, \code{qawo}, \code{qawf}, \code{qawc}, and
\code{qaws} of \code{QUADPACK}.  Which of these integrators is called
depends on the weight used and the options passed to the function
\code{quad} which has the following definition:
\begin{verbatim}
quad(func,a,b,args=(),full_output=0,epsabs=1.49e-8,epsrel=1.49e-8,limit=50,
     points=None,weight=None,wvar=None,wopts=None,maxp1=50,limlst=50)
\end{verbatim}

\subsubsection{Typical Usage}
Under typical usage, one specifies the function, the limits of
integration and possibly additional arguments to the
function.
\begin{verbatim}
>>> result,abserr = quad(func, a, b, extra_args)
\end{verbatim}
Function \code{quad} returns the result of the integration in
\code{result} and an absolute error estimate in \code{abserr}.  If $I$
is the true integral, $R$ is the computed result, and $E$ is the
error, then it should be that $|I-R| \leq E$.  If there is an input
error or a Python error occurs (likely due to calling the function
incorrectly) an exception is raised.  If a numerical problem is
encountered while integrating the function a warning message is
displayed explaining the problem but the results are still
returned. (See \code{full\_output} for alternate behavior.)

\subsubsection{Input Arguments}

The behind-the-scenes code of \code{quad} can vary quite a bit depending
on the input arguments which are now explained.
\begin{description}
\item[\code{func}] --- The function, $f$, to be integrated.  This should be
a callable object whose first argument is the variable of integration.
The callable object should treat the first argument as a Float.
Additional arguments can be passed to the function with \code{args}. 
\item[\code{(a,b)}] --- The limits of integration.  Use Multipack.Inf
and -Multipack.Inf for infinite limits. Infinite limits are only
available with no weight or a sinusoidal weight.  
\item[\code{args}] --- Extra arguments to pass to the integrand.
If only a single extra argument is needed simply pass it here.  If
multiple extra arguments are needed wrap them in a tuple.
\item[\code{full\_output}] --- A flag indicating whether or not a
dictionary of optional output is desired as the third output.  If
\code{full\_output=1} the warning message regarding difficult
integrations is suppressed and appended to the return tuple.  Some
contents of the dictionary vary depending on the type of integration
performed.  It is useful to know that for infinite limits the range is
transformed to $(0,1)$ and the optional outputs are given with respect
to this transformed range. The common keys of this dictionary which
we'll call \code{info} are: (Let $M=$ \code{limit}.)
\begin{description}
\item[\code{'neval'}] --- The number of function evaluations.
\item[\code{'last'}] --- The number of subintervals produced in the
subdivision process.  Call this number $K$.
\item[\code{'alist'}] --- A rank-1 array of length $M$, the first $K$
elements of which are the left end points of the subintervals in the
partition of the integration range.  
\item[\code{'blist'}] --- A rank-1 array of length $M$, the first $K$
elements of which are the right end points of the subintervals. 
\item[\code{'rlist'}] --- A rank-1 array of length $M$, the first $K$
elements of which are the integral approximations on the subintervals.
\item[\code{'elist'}] --- A rank-1 array of length $M$, the first $K$
elements of which are the moduli of the absolute error estimates on
the subintervals.
\item[\code{'iord'}] --- A rank-1 integer array of length $M$, the first $L$
elements of which are pointers to the error estimates over the
subintervals with $L = K$ if $K \leq M/2 + 2$ or $L = M+1-K$
otherwise.  Let $I$ be the sequence \code{info['iord']} and let $E$ be
the sequence \code{info['elist']}.  Then $E_{I_1},\ldots,E_{I_L}$
forms a decreasing sequence.
\end{description}
\item[\code{epsabs}] --- Absolute accuracy requested $>0$.  
\item[\code{epsrel}] --- Relative accuracy requested $\geq
\textrm{max}(50*\epsilon, 0.5 \times 10^{-28})$.
\item[\code{limit}] --- An upper bound on the number of
subintervals in the partition of the integration region.
\item[\code{points}] --- A sequence (call its length $P$) of user
provided break points of the (finite) integration interval where local
difficulties of the integrand may occur ({\em e.g.} singularities,
discontinuities).  The sequence does not have to be sorted. If this
sequence is provided the following additional outputs are placed in
the optional output dictionary.  
\begin{description}
\item[\code{'pts'}] --- A rank-1 array of length $P+2$ containing the
integration limits and the break points of the interval in ascending order. 
\item[\code{'level'}] --- A rank-1 integer array of length $M$
(=\code{limit}), containing the subdivision levels of the subintervals,
{\em i.e.} if $(aa,bb)$ is a subinterval of $(p_1,p_2)$ where $p_1$ and
$p_2$ are adjacent elements of \code{info['pts']}, then $(aa,bb)$ has
level $l$ if $|bb-aa| = |p_2-p_1| 2^{-l}.$ 
\item[\code{'ndin'}] --- A rank-1 integer array of length $P+2$.
After first integration over the intervals $(p_1,p_2)$, the error
estimates over some of the intervals may have been increased
artificially, in order to put their subdivision forward.  This array
has ones in slots corresponding to the subintervals for which this
happens.  All other values are 0.
\end{description}
\item[\code{weight}] --- A string indicating a weighting function
with which to multiply the integrand over the integration region. One
of the following: \code{'cos'}, \code{'sin'}, \code{'alg'},
\code{'alg-loga'}, \code{'alg-logb'}, \code{'alg-log'},
\code{'cauchy'}.  See below for additional information.
\item[\code{wvar}] --- Variable(s) to pass to weighting function.
\item[\code{wopts}] --- Optional inputs for reusing Chebyshev
moments when performing multiple integrations using Clenshaw-Curtis
method with a sinusoidal weight over the same finite interval.  When
used, the first element of this length 2 sequence is \code{info['momcom']},
the second element is \code{info['chebmo']} where \code{info} is the
optional dictionary from a previous run over the same interval.
\item[\code{maxp1}] --- An upper bound on the number of Chebyshev
moments that can be stored, {\em i.e.} if $M_p=$\code{info['maxp1']},
then moments can only be stored for intervals of lengths $|b-a|
2^{-l}, l=0,1,\ldots, M_p-1$.  $M_p \geq 1$.
\item[\code{limlst}] --- An upper bound on the number of cycles
$\geq 3$.  Only used for sinusoidal weighting with an infinite end-point. 
\end{description}

\subsubsection{Weighting functions: inputs and optional outputs}
The following weighting functions are available.  None of these can be
used with infinite end points except for the \code{'sin'} or
\code{'cos'} weighting which can use a single infinite end-point.
\begin{description}
\item[\code{'cos','sin'}] --- $w(x) = \cos(\omega x)$ or $w(x) =
\sin(\omega x)$.  Set \code{wvar} to $\omega$.  

\textbf{Finite Integration Limits}

If both integration limits are finite then the integration is
performed using a Clenshaw-Curtis method which uses Chebyshev moments.
If \code{full\_output} is 1, then these moments are provided through entries
in the info dictionary which has additional members:
\begin{description}
\item[\code{'momcom'}] --- The maximum level of Chebyshev moments that 
been computed, {\em i.e.} if $M_c$ is
\code{info['momcom']} then the moments have been computed for
intervals of length $|b-a|2^{-l}, l=0,1,\ldots,M_c$.
\item[\code{'nnlog'}] --- A rank-1 integer array of length $M$
(=\code{limit}), containing the subdivision levels of the
subintervals, {\em i.e.} an element of this array is equal to $l$ if
the corresponding subinterval is of length $|b-a| 2^{-l}$.
\item[\code{'chebmo'}] --- A rank-2 array of shape (25,\code{maxp1})
containing the computed Chebyshev moments.  These can be passed on to
an integration over the same interval by passing this array as the
second element of the sequence \code{wopts} and passing
\code{info['momcom']} as the first element.
\end{description} 

\textbf{Infinite Integration Limits}

If one of the integration limits is infinite ($\pm$Multipack.Inf) then
a Fourier integral is computed (assuming $\omega \neq 0$).  If
\code{full\_output} is 1 and a numerical error is encountered, besides
the error message attached to the output tuple, an dictionary is also
appended to the output tuple which translates the error codes in the array
\code{info['ierlst']} to English.  The output information dictionary
contains the following entries instead of \code{'last'},
\code{'alist'}, \code{'blist'}, \code{'rlist'}, and \code{'elist'}.
\begin{description}
\item[\code{'lst'}] --- The number of subintervals needed for the
integration call it $K_f$ in what follows.
\item[\code{'rslst'}] --- A rank-1 array of length
$M_f=$\code{limlst}, whose first $K_f$ elements contain the integral
contribution over the interval $(a + (k-1)c, a+kc)$ where $c =
\left(2\left\lfloor\left| \omega \right| \right\rfloor+1\right) \pi /
\left| \omega \right|$ where $k = 1,2,\ldots,K_f$.
\item[\code{'erlst'}] --- A rank-1 array of length $M_f$ containing
the error estimate corresponding to the interval in the same
positition in \code{info['rslst']}. 
\item[\code{'ierlst'}] --- A rank-1 integer array of length $M_f$
containing an error flag corresponding to the interval in the same
postion in \code{info['rslst']}.  See the explanation dictionary (last
output) for the meaning of the codes.
\end{description}

\item[\code{'alg'}] --- $w(x) = (x-a)^\alpha (b-x)^\beta$.  Set
\code{wvar} to $(\alpha,\beta)$.  
\item[\code{'alg-loga'}] --- $w(x) = (x-a)^\alpha (b-x)^\beta
\textrm{log}(x-a)$. Set \code{wvar} to $(\alpha, \beta)$.
\item[\code{'alg-logb'}] --- $w(x) = (x-a)^\alpha (b-x)^\beta
\textrm{log}(b-x)$. Set \code{wvar} to $(\alpha, \beta)$.
\item[\code{'alg-log'}] --- $w(x) = (x-a)^\alpha (b-x)^\beta
\textrm{log}(x-a) \textrm{log}(b-x)$. Set \code{wvar} to $(\alpha,
\beta)$. 

\item[\code{'cauchy'}] --- $w(x) = 1/(x-c)$.  Set \code{wvar} to $c$.
\end{description}

\subsection{Multivariate Integration}

Some effort was taken to allow the function \code{quad} to be called
recursively.  That is, the function to be integrated can itself call
\code{quad}.  This allows for the construction of simple (and naive) N-variate
integrators rather straightforwardly.  As an illustration, both a
double and triple integration function are provided in
\code{Multipack} (\code{quadpack}).  These multivariate integrators do
not currently take advantage of the advanced options of \code{quad}.

\subsubsection{Double Integrals}

Suppose we wish to find a numerical approximation to the definite
double integral
\[
\int_a^b \int_{g(x)}^{h(x)} f(x,y) dy dx
\]
We can use the function \code{dblquad} to do this.  This function must
be called (with defaults for keyword arguments shown) as
\begin{verbatim}
dblquad(func2d,a,b,gfun,hfun,extra_args=(),epsabs=1.49e-8,epsrel=1.49e-8):
\end{verbatim}
This function always returns a tuple consisting of the result as the
first element and an estimate of the error over the outermost integral
(assuming the inner integrals are computed exactly) as the second
element. The arguments to \code{dblquad} are defined as follows.
\begin{description}
\item[\code{func2d}] --- The bivariate function to be integrated.  This
must be defined to take at least two arguments in the order $(y,x)$.
\textbf{Note that this is opposite from the natural definition.}  If
further arguments to the function are necessary they can be passed
using the input \code{extra\_args}.
\item[\code{(a,b)}] --- The limits of integration over the $x$ variable.
\item[\code{gfun}] --- A function which takes a single float argument ($x$)
and returns a float as a result.  It is the lower limit of integration
over the $y$ variable.  (Note that \code{lambda} can be useful
here as with entering the other function describing the region of
integration).
\item[\code{hfun}] --- A single-variable, single return function
representing the upper limit of integration over the $y$ variable.
\item[\code{extra\_args}] --- Extra arguments to pass to \code{func}
in a tuple or singly if just one.
\end{description}
Note that the arguments \code{epsabs} and \code{epsrel} are passed
directly on to the univariate integration steps and therefore
represent desired error only over the (recursively called) 1-D
integrals.  They do not represent the total error desired over the
entire double integral. 

\subsubsection{Triple Integrals}

Suppose we want to find an approximation to the definite triple
integral
\[
\int_a^b \int_{g(x)}^{h(x)} \int_{q(x,y)}^{r(x,y)} f(x,y,z) dz dy dx.
\]
The function \code{tplquad} can accomplish this.  It's definition is
as follows.
\begin{verbatim}
tplquad(func3d,a,b,gfun,hfun,qfun,rfun,extra_args=(),epsabs=1.49e-8,
        epsrel=1.49e-8)
\end{verbatim}
This is very similar to the function \code{dblquad} with the following
differences.
\begin{description}
\item[\code{func3d}] --- The three-dimensional function to be
integrated.  This function still must return a float but must now take
three arguments in the order $(z,y,x)$.  Extra arguments can still be
passed through \code{extra\_arguments}
\item[\code{qfun}] --- This defines the lower surface of integration
for the $z$ variable.  It must be a function the takes two floats in
the order $(x,y)$ and returns a float.  Note the ``normal'' order of
the input variables.
\item[\code{rfun}] --- This defines the upper surface of integration
over the $z$ variable. It also must be a function that takes two
floats in the order $(x,y)$ and returns a float.  Note the ``normal''
order of the input variables.
\end{description}



%%% Local Variables: 
%%% mode: latex
%%% TeX-master: "main"
%%% TeX-master: "main"
%%% End: 






