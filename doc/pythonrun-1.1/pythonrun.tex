\documentclass[a4paper,12pt]{article}
\usepackage{a4wide}
\usepackage{amsmath}
\usepackage{pythonrun}
\newcommand{\bs}{\symbol{`\\}}
%%tth:\newenvironment{pmatrix}{\left(\begin{array}1}{\end{array}\right)}
\ifpythondistil\renewcommand{\bs}{}\fi
\title{\texttt{pythonrun.sty v1.1} User's Guide\\
\texttt{\normalsize http://koer.ioc.ee/\~{}pearu/python/}}
\author{Pearu Peterson\\\texttt{\normalsize<pearu@ioc.ee>}}

\begin{document}

\maketitle

\begin{abstract}
  The \texttt{pythonrun.sty} is a LaTeX2e package that can be used for
  writing a Latex document if the author wishes to include python code
  fragments to it. Most importantly, these code fragments can be
  executed and the results of can be included to Latex output,
  dynamically. These Latex documents must be compiled with
  \texttt{--shell-escape} option. One can also ``distill'' these Latex
  documents (that is to make them ``static'', and so that the
  \texttt{latex->html}  converters can be applied).
  
  The purpose of \texttt{pythonrun.sty} is to ease writing python
  specific tutorials in LaTeX.
\end{abstract}
\ifpythondistil\else
\ifshellescape\else
\begin{center}
\large NB! Use \verb+latex --shell-escape+ for compiling this LaTeX
document in order to get a complete version.  
\end{center}
\fi\fi

\tableofcontents


\section{Representing Python code in LaTeX}
\label{sec:repr-pyth-code}

Let us first create a python file from a latex document. For that we do
\begin{verbatim}
\begin{python}
from Numeric import *
\end{python}
\pythonsave{init.py}
\pythonreset
\end{verbatim}
(see the following sections to learn what is going on here).
\begin{python}
from Numeric import *
\end{python}
\pythonsave{init.py}
\pythonreset

\subsection{Commands \texttt{\bs pythoninput} and \texttt{\bs pythoninput*}}

You can read a python code fragment from a file using \verb+\pythoninput+ 
command. For example, 
\begin{verbatim}
\pythoninput{init.py}
\end{verbatim}
inserts the context of the file \verb+init.py+ to LaTeX document
showing it in \verb+verbatim+ environment and saves it for latter use
in python. Below is shown the context
of the file \verb+init.py+ after using this command:%
\pythoninput{init.py}

If you wish to just load a python file but not show it in latex
output, use
\begin{verbatim}
\pythoninput*{init.py}
\end{verbatim}
instead.
\pythonreset           % This is to make sure that everything works
\pythoninput*{init.py} % correctly
                       
\subsection{Environments \texttt{python} and \texttt{python*}}

You can give a python code fragment in the \verb+python+ environment. For example,
\begin{verbatim}
\begin{python}
def F(x):
    return 1-x*x
\end{python}
\end{verbatim}
shows it in verbatim
\begin{python}
def F(x):
    return 1-x*x
\end{python}
and saves it for future use in python. 

Use \verb+python*+ environment to hide the context from the latex but
not from the python. For example, 
\begin{verbatim}
\begin{python*}
def G(x):
    return 1+x*x
\end{python*}
\end{verbatim}
defines a python function \verb+G+ that can be used later but this
definition is not shown in latex output.
\begin{python*}
def G(x):
    return 1+x*x
\end{python*}


\section{Calling Python from LaTeX}

Now that we have some python code fragments, we are ready to
demonstrate how to call Python on these fragments. There are number of
ways to do it using \verb+pythonrun.sty+ and the choice depends on the
effect that user desires.
The main principle is the following
\begin{quotation}
Collect all saved python code fragments into one fragment and
  call python on it.
\end{quotation}
Now depending on whether to show and/or to save the last python code
fragment, one should use different commands/environments.

\subsection{Environments \texttt{pythonrun} and \texttt{pythonrun*}}

To show the current fragment of python code in addition to running
python, use environment 
\verb+pythonrun+. For example,
\begin{verbatim}
\begin{pythonrun}
print F(2),G(2)
\end{pythonrun}
\end{verbatim}
shows the python code and the results of the python call on the collected
python code fragments as shown below:
\begin{pythonrun}
print F(2),G(2)
\end{pythonrun}

To show only the results, use environment \verb+pythonrun*+ instead. For
example,
\begin{pythonrun*}
print F(3),G(3)
\end{pythonrun*}
shows the results of the python call \verb+print F(3),G(3)+.

The environments above will not save python code fragments for future
use. In order to force it, use environments \verb+pythonrunwithsave+
and \verb+pythonrunwithsave*+, respectively. However, one should need
them only in rare cases. For example,
\begin{verbatim}
\begin{pythonrunwithsave}
print 'Hello world!'
\end{pythonrunwithsave}
\end{verbatim}
causes the printout \verb+Hello world!+ in all of the following calls
to python.
\begin{pythonrunwithsave}
print 'Hello world!'
\end{pythonrunwithsave}
To see it explicitly, let's call python with
\begin{pythonrun}
print F(4),G(4)
\end{pythonrun}


\section{Saving, resetting, cleaning up, and unsuccessful calls}

\subsection{Command \texttt{\bs pythonsave}}

Command \verb+\pythonsave+ takes a file name as an argument and it
collects all saved python code fragments and puts them into given
file. See, for example, the creation of the file \verb+init.py+ in
Section~\ref{sec:repr-pyth-code}.

\subsection{Command \texttt{\bs pythonreset}}

Command \verb+\pythonreset+ can be used for starting a new ``python
session''. So, after calling
\begin{verbatim}
\pythonreset
\end{verbatim}
all python code fragments are forgotten and one can start inserting
new ones.
\pythonreset

\subsection{Command \texttt{\bs pythonclean}}

It is recommended to put \verb+\pythonclean+ at the end of latex
document. It then removes all auxiliary python files created by
\verb+pythonrun.sty+.

\subsection{Unsuccessful python calls}

Note that after resetting the function \verb+F+ is not defined (we are
in a new python session). Here is what happens if one ignores it and
causes the error in python call:
\begin{pythonrun}
print F(7)
\end{pythonrun}
Errors are handled in this way only in environments \verb+pythonrun+
and \verb+pythonrun*+. Using environments \verb+pythoncall+,
\verb+pythoncall*+, or command \verb+\pythonline+ (see below), the
errors may transfer over to latex. If this happens, one can use
\verb+\pythonsave+ to save python code fragments to a file and test the
code directly.

One can use also \verb+\ifshellescape \else \fi+ construction for
debugging. 

\section{Representing results in LaTeX}

\subsection{Environment \texttt{pythoncall}}

Environment \verb+pythonrun+ (and its relatives) shows the results of
the python call in verbatim environment. This is safe even if errors
occur. However, one may need to format the results using LaTeX
features. For that, user can use the environment
\verb+pythoncall+. For example, let us define (again) functions
\begin{python}
def F(x):
    return 1-x*x
def G(x):
    return 1+x*x
\end{python}
and writing in latex
\begin{verbatim}
\begin{displaymath}
F(2),G(2)=
\begin{pythoncall}
print F(2),',',G(2)
\end{pythoncall}
\end{displaymath}
\end{verbatim}
we will have the following latex output
\begin{displaymath}
F(2),G(2)=
\begin{pythoncall}
print F(2),',',G(2)
\end{pythoncall}
\end{displaymath}

\subsection{Command \texttt{\bs pythonline}}

Command \verb+\pythonline+ is a command version of the environment
\verb+pythoncall+. For example,
$F(5)/G(5)=\pythonline{print F(5)}/%
\pythonline{print G(5)}\approx
\pythonline{print round(F(5)/float(G(5)),6)}$ 
is obtained as a result of
\begin{verbatim}
$F(5)/G(5)=\pythonline{print F(5)}/\pythonline{print G(5)}\approx
\pythonline{print round(F(5)/float(G(5)),6)}$ 
\end{verbatim}

But be careful when using multiple lines in \verb+pythonline+
argument! The output may not be what expected (because Python is very
strict when dealing with spaces, tabs, newlines, etc).


\section{Examples}

First we reset
\pythonreset
and then introduce some python code
\begin{python}
import string; strrep=string.replace
from Numeric import *
def format4latex(x):
    try: x=x.tolist()
    except: x=list(x)
    s=strrep(strrep(strrep(`x`,' ',''),'],[','\\\\'),',','&')
    while s[0]=='[': s=s[1:-1]
    return s
def pmatrix(x):
    print '\\begin{pmatrix}'+format4latex(x)+'\\end{pmatrix}'
\end{python}
Let us consider two matrices in python:
\begin{python}
A=array([[1,2],[3,4]])
B=transpose(A)
\end{python}
or in latex:
\begin{displaymath}
  A=\pythonline{pmatrix(A)},\quad
  B=A^T=\pythonline{pmatrix(B)},\quad\Rightarrow\qquad
  A*B=\pythonline{pmatrix(matrixmultiply(A,B))}.
\end{displaymath}
This is a result of the following
\begin{verbatim}
\begin{displaymath}
  A=\pythonline{pmatrix(A)},\quad
  B=A^T=\pythonline{pmatrix(B)},\quad\Rightarrow\qquad
  A*B=\pythonline{pmatrix(matrixmultiply(A,B))}.
\end{displaymath}
\end{verbatim}


\section{Notes}

\subsection{Using Emacs}


When I was writing this document in emacs, I used the following
commands in my \texttt{.emacs} file:
\begin{verbatim}
(global-set-key [f7] 'latex-mode)
(global-set-key [f8] 'python-mode)
\end{verbatim}
These were quite useful when editing this rather mixed latex-python
document.

\subsection{Distillation}

Starting with the version \verb+1.1+, \verb+pythonrun.sty+ comes with
a Python program \verb+dpythonrun.py+. This program translates users
Latex document (that uses \verb+pythonrun.sty+) to its ``static''
version. In this static document the use of \verb+pythonrun.sty+ is not
required anymore and it contains also the results of the python calls. 
This new document can be used for converting it to some other
format, say, to HTML.

The usage of \verb+dpythonrun.py+ is the following. It takes two optional
arguments: an input file name as a first argument and an output file
name as a second argument (if the latter is missing, the output file name
is constructed by prepending \verb+distil_+ to the input file name).
Here is an example how the current User's Guide was converted to HTML
using \verb+tth+\footnote{See
  \texttt{http://hutchinson.belmont.ma.us/tth/}}:
\begin{verbatim}
$ dpythonrun.py pythonrun.tex # This generates distil_pythonrun.tex
$ latex distil_pythonrun.tex  # Three times
$ tth < distil_pythonrun.tex > pythonrun.html -Ldistil_pythonrun -i
\end{verbatim}

%Let us clean up
\pythonclean
\ifpythondistil\else
\ifshellescape\else
\typeout{***********************************************************}
\typeout{Latex this document with --shell-escape option in order to }
\typeout{get a complete version:}
\typeout{ latex --shell-escape pythonrun.tex}
\typeout{***********************************************************}
\fi\fi
\end{document}

%%% Local Variables: 
%%% mode: latex
%%% TeX-master: t
%%% End: 
